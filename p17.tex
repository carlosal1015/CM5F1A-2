\section{Pregunta N$^{\circ}$17\qquad Carlos Alonso Aznarán Laos}

\begin{frame}[fragile]
    \begin{enumerate}\setcounter{enumi}{16}
        \item

              Use el método de la transformada de Fourier discreta
              para calcular el polinomio trigonométrico interpolante
              de quinto grado en $\left[-\pi,\pi\right]$ para
              \begin{math}
                  f\left(x\right)=
                  \dfrac{
                      \cos\left(2\pi x\right)+
                      \sin\left(2\pi x\right)
                  }{x^{2}+1}
              \end{math},
              donde
              \begin{math}
                  x_{j}=
                  -\pi+
                  \dfrac{j\pi}{4}
              \end{math},
              \begin{math}
                  j=0,1,2,\dotsc,7
              \end{math}
              y
              \begin{math}
                  y_{j}=
                  f\left(x_{j}\right)
              \end{math}.
    \end{enumerate}

    \begin{solution}
\begin{verbatim}
def DFT(x):
    """ Calculates la transformada discreta de Fourier 1D de un vector.
    """
    n = len(x)
    y = [0]*n
    omega = np.exp(-2.0j*np.pi/n)
    for k in range(0,n):
        y[k] = np.sum(x*omega**(np.arange(0,n)*k))
    return y
\end{verbatim}
    \end{solution}
\end{frame}

