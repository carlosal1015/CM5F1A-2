\section{Pregunta N$^{\circ}$6\qquad Andre Gilmer Santos Felix}

\begin{frame}
    \begin{enumerate}\setcounter{enumi}{5}
        \item

              Use
              \begin{math}
                  F\left(x\right)=
                  \cos\left(2x\right)
              \end{math}
              para obtener la aproximación de mínimos cuadrados
              discretos
              \begin{math}
                  S_{2}\left(x\right)
              \end{math}
              para los datos
              \begin{math}
                  \left\{
                  \left(x_{j},y_{j}\right)
                  \right\}_{j=0}^{7}
              \end{math},
              donde
              \begin{math}
                  x_{j}=
                  \dfrac{\pi j}{m}-
                  \pi
              \end{math}
              y
              \begin{math}
                  y_{j}=
                  F\left(x_{j}\right)
              \end{math}
              en el intervalo
              \begin{math}
                  \left[
                      -\pi,
                      \pi
                      \right]
              \end{math}.
    \end{enumerate}

    \begin{solution}
        Sea el polinomio trigonométrico de grado $2$
        \begin{equation*}
            S_{2}\left(x\right)=
            \dfrac{a_{0}}{2}+
            a_{2}\cos(2x)+
            a_{1}\cos\left(x\right)+
            b_{1}\operatorname{sen}\left(x\right),
        \end{equation*}
        donde
        \begin{align*}
            a_{k} & =
            \dfrac{1}{6}
            \sum_{j=1}^{11}
            y_{j}\cos\left(kx_{j}\right),\quad
            \forall k\in\left\{0,\dotsc,n-1\right\} \\
            b_{k} & =
            \dfrac{1}{6}
            \sum_{j=1}^{11}
            y_{j}
            \operatorname{sen}\left(kx_{j}\right),\quad
            \forall k\in\left\{0,\dotsc,n-1\right\}.
        \end{align*}
    \end{solution}
\end{frame}